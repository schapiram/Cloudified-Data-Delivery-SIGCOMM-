\documentclass{hotnets17}

\usepackage{times}  
\usepackage{hyperref}

\hypersetup{pdfstartview=FitH,pdfpagelayout=SinglePage}

\setlength\paperheight {11in}
\setlength\paperwidth {8.5in}
\setlength{\textwidth}{7in}
\setlength{\textheight}{9.25in}
\setlength{\oddsidemargin}{-.25in}
\setlength{\evensidemargin}{-.25in}

\begin{document}

% \conferenceinfo{HotNets 2017} {}
% \CopyrightYear{2017}
% \crdata{X}
% \date{}

%%%%%%%%%%%% THIS IS WHERE WE PUT IN THE TITLE AND AUTHORS %%%%%%%%%%%%

\title{HotNets 2017 Paper}

\author{Paper \#0, 3 pages}

\maketitle

%%%%%%%%%%%%%  ABSTRACT GOES HERE %%%%%%%%%%%%%%
\begin{abstract}

Everybody loves TCP~\cite{vanjacobson}. The paper body in this example
using the HotNets 2017 style file contains two copies
of the CFP text to show the correct format of a standard text page.

\end{abstract}

\section{Call for Papers}

The 16th ACM Workshop on Hot Topics in Networks (HotNets 2017) will
bring together researchers in computer networks and systems to engage
in a lively debate on the theory and practice of computer
networking. HotNets provides a venue for discussing innovative ideas
and for debating future research agendas in networking.

We invite researchers and practitioners to submit short position
papers. We encourage papers that identify fundamental open questions,
advocate a new approach, offer a constructive critique of the state of
networking research, re-frame or debunk existing work, report
unexpected early results from a deployment, report on promising but
unproven ideas, or propose new evaluation methods. Novel ideas need
not be supported by full evaluations; well-reasoned arguments or
preliminary evaluations can support the possibility of the paper's
claims. We seek early-stage work, where the authors can benefit from
community feedback. An ideal submission has the potential to open a
line of inquiry for the community that results in multiple conference
papers in related venues (SIGCOMM, SOSP, OSDI, NSDI, MobiCom, MobiSys,
CoNEXT, etc.), rather than a single follow-on conference paper. The
program committee will explicitly favor early work and papers likely
to stimulate reflection and discussion over ``conference papers in
miniature.'' Finished work that fits in a short paper is likely a
better fit with the short-paper tracks at either CoNEXT or IMC.

HotNets takes a broad view of networking research. This includes new
ideas relating to (but not limited to) mobile, wide-area, data-center,
home, and enterprise networks using a variety of link technologies
(wired, wireless, visual, and acoustic), as well as social networks
and network architecture. It encompasses all aspects of networks,
including (but not limited to) packet-processing hardware and
software, virtualization, mobility, provisioning and resource
management, performance, energy consumption, topology, robustness and
security, measurement, diagnosis, verification, privacy, economics and
evolution, interactions with applications, and usability of underlying
networking technologies.

Position papers will be selected based on originality, likelihood of
stimulating insightful discussion at the workshop, and technical
merit. Accepted papers will be posted online prior to the workshop and
will be published in the ACM Digital Library to facilitate wide
dissemination of the ideas discussed at the workshop.

\section{New in 2017}

Toward facilitating reflective discussion within the networking
research community, HotNets 2017 will introduce {\it HotNets Dialogues:}
give-and-take written discussions of a selection of accepted papers,
each between a pair of program committee members, that will be posted
on the HotNets web site shortly before the workshop.

\section{Concurrent Submission Policy}

Concurrent submissions to HotNets 2017 and any other peer-reviewed
venue that cover the same work (differences in degree of detail given
the two venues' length limits notwithstanding) are prohibited, and
will result in the rejection of the HotNets submission in
question. ``Concurrent'' means any other peer-reviewed venue whose
reviewing period (i.e., between submission and notification) overlaps
with that of HotNets. The ``same work'' means, for example, a submission
overlapping significantly in content with a conference
submission. However, a position paper submitted to HotNets (e.g., that
reflects broadly on the state of some aspect of the field, adopts a
position as to how the field should move forward, or articulates a
broad avenue of future work) will not be considered the ``same work'' as
a conference-length paper on a specific system that addresses a point
under the broad umbrella covered by the position paper. Authors with
questions about the concurrent submission policy are encouraged to
contact the PC chairs prior to submitting.

\section{Workshop Participation}

HotNets attendance is limited to roughly 90 people, to facilitate
lively discussion. Invitations will be allocated first to one author
of each paper, HotNets organizers and committee members, and
conference sponsors. To promote an inclusive workshop, HotNets will
also make a limited number of open registration slots available to the
community.

\section{Submission and Formatting}

Submitted papers must be no longer than 6 pages (10 point font, 12
point leading, 7 inch by 9.25 inch text block) including all content
except references. Authors may use up to one further page beyond these
6 for references. Templates and a link to the submission web site will
be posted nearer to the submission deadline.

All submissions must be blind: they must not indicate the names or
affiliations of the authors in the paper. Only electronic submissions
in PDF will be accepted. Submissions must be written in English,
render without error using standard tools (e.g., Acrobat Reader), and
print on US Letter paper. Papers must contain novel ideas and must
differ significantly in content from previously published papers.

\section{Important Dates}

{
\small 
\begin{tabular}{ll}
Paper submission: 	        & August 4, 2017 (7:59PM EDT)\\
Workshop dates: 	        & November 30--December 1, 2017\\
\end{tabular}
}

\section{Call for Papers}

The 16th ACM Workshop on Hot Topics in Networks (HotNets 2017) will
bring together researchers in computer networks and systems to engage
in a lively debate on the theory and practice of computer
networking. HotNets provides a venue for discussing innovative ideas
and for debating future research agendas in networking.

We invite researchers and practitioners to submit short position
papers. We encourage papers that identify fundamental open questions,
advocate a new approach, offer a constructive critique of the state of
networking research, re-frame or debunk existing work, report
unexpected early results from a deployment, report on promising but
unproven ideas, or propose new evaluation methods. Novel ideas need
not be supported by full evaluations; well-reasoned arguments or
preliminary evaluations can support the possibility of the paper's
claims. We seek early-stage work, where the authors can benefit from
community feedback. An ideal submission has the potential to open a
line of inquiry for the community that results in multiple conference
papers in related venues (SIGCOMM, SOSP, OSDI, NSDI, MobiCom, MobiSys,
CoNEXT, etc.), rather than a single follow-on conference paper. The
program committee will explicitly favor early work and papers likely
to stimulate reflection and discussion over ``conference papers in
miniature.'' Finished work that fits in a short paper is likely a
better fit with the short-paper tracks at either CoNEXT or IMC.

HotNets takes a broad view of networking research. This includes new
ideas relating to (but not limited to) mobile, wide-area, data-center,
home, and enterprise networks using a variety of link technologies
(wired, wireless, visual, and acoustic), as well as social networks
and network architecture. It encompasses all aspects of networks,
including (but not limited to) packet-processing hardware and
software, virtualization, mobility, provisioning and resource
management, performance, energy consumption, topology, robustness and
security, measurement, diagnosis, verification, privacy, economics and
evolution, interactions with applications, and usability of underlying
networking technologies.

Position papers will be selected based on originality, likelihood of
stimulating insightful discussion at the workshop, and technical
merit. Accepted papers will be posted online prior to the workshop and
will be published in the ACM Digital Library to facilitate wide
dissemination of the ideas discussed at the workshop.

\section{New in 2017}

Toward facilitating reflective discussion within the networking
research community, HotNets 2017 will introduce {\it HotNets Dialogues:}
give-and-take written discussions of a selection of accepted papers,
each between a pair of program committee members, that will be posted
on the HotNets web site shortly before the workshop.

\section{Concurrent Submission Policy}

Concurrent submissions to HotNets 2017 and any other peer-reviewed
venue that cover the same work (differences in degree of detail given
the two venues’ length limits notwithstanding) are prohibited, and
will result in the rejection of the HotNets submission in
question. ``Concurrent'' means any other peer-reviewed venue whose
reviewing period (i.e., between submission and notification) overlaps
with that of HotNets. The ``same work'' means, for example, a submission
overlapping significantly in content with a conference
submission. However, a position paper submitted to HotNets (e.g., that
reflects broadly on the state of some aspect of the field, adopts a
position as to how the field should move forward, or articulates a
broad avenue of future work) will not be considered the ``same work'' as
a conference-length paper on a specific system that addresses a point
under the broad umbrella covered by the position paper. Authors with
questions about the concurrent submission policy are encouraged to
contact the PC chairs prior to submitting.

\section{Workshop Participation}

HotNets attendance is limited to roughly 90 people, to facilitate
lively discussion. Invitations will be allocated first to one author
of each paper, HotNets organizers and committee members, and
conference sponsors. To promote an inclusive workshop, HotNets will
also make a limited number of open registration slots available to the
community.

\section{Submission and Formatting}

Submitted papers must be no longer than 6 pages (10 point font, 12
point leading, 7 inch by 9.25 inch text block) including all content
except references. Authors may use up to one further page beyond these
6 for references. Templates and a link to the submission web site will
be posted nearer to the submission deadline.

All submissions must be blind: they must not indicate the names or
affiliations of the authors in the paper. Only electronic submissions
in PDF will be accepted. Submissions must be written in English,
render without error using standard tools (e.g., Acrobat Reader), and
print on US Letter paper. Papers must contain novel ideas and must
differ significantly in content from previously published papers.

\section{Important Dates}

{
\small 
\begin{tabular}{ll}
Paper submission: 	        & August 4, 2017 (7:59PM EDT)\\
Workshop dates: 	        & November 30--December 1, 2017\\
\end{tabular}
}

\section*{Acknowledgments}

\bibliographystyle{abbrv} 
\begin{small}
\bibliography{hotnets17}
\end{small}

\end{document}

